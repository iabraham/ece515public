\subsection{Homework 2 - Due: 09/11}\label{homework-2---due-0911}

\subsubsection{Problem 1}\label{problem-1}

Convert each of the following higher-order differential equations into
state-space form:

\begin{itemize}
\tightlist
\item
  System (a): \[\ddot x = \dot u + 3 u\]
\item
  System (b): \[\dfrac{d^3 x}{dt^3} - 2 \dfrac{d^2x}{dt^2} + x = u\]
\item
  System (c): \[\ddot x = \sin \left( \dot x - x \right)\]
\end{itemize}

\subsubsection{Problem 2}\label{problem-2}

\begin{enumerate}
\def\labelenumi{\alph{enumi}.}
\tightlist
\item
  Let \(A\) be an \(m \times n\) matrix. Show that if there exists a
  matrix \(B\) such that \(BA=I_{n}\) then the columns of \(A\) are
  linearly independent.
\item
  Next show that if there exists a matrix \(C\) such that \(AC=I_{m}\)
  then the rows of \(A\) are linearly independent.
\item
  Matrices \(B\) and \(C\) (when they exist) are called left and right
  inverses of \(A\). Prove that if \(A\) has a left and a right inverse
  then \(A\) is square and the left and right inverses must be equal.
\item
  Suppose for some matrices \(X\) and \(Y\) we have that \(AX = AY\). Is
  it true that \(X=Y\)? Prove or provide a counter example.
\end{enumerate}

\subsubsection{Problem 3}\label{problem-3}

Let \(A: X \to Y\) be a linear transformation.

\begin{enumerate}
\def\labelenumi{(\alph{enumi})}
\tightlist
\item
  Prove that \(\dim N (A) + \dim R(A) = \dim X\) (the sum of the
  dimension of the nullspace of \(A\) and the dimension of the range of
  \(A\) equals the dimension of \(X\)).
\item
  Now assume that \(X = Y\). It is not always true that \(X\) is a
  \href{exlecs/lec02.html\#def-directsum}{direct sum} of \(N(A)\) and
  \(R(A)\). Find a counterexample demonstrating this. Also, describe a
  class of linear transformations (as general as you can think of) for
  which this statement is true.
\end{enumerate}

\subsubsection{Problem 4}\label{problem-4}

Let \(A\) be a real valued \(n \times n\) matrix. Suppose that
\(\lambda + i \mu\) is a complex eigenvalue of \(A\) and \(x + iy\) is a
corresponding complex eigenvector. Here \(\lambda, \mu \in \mathbb{R}\)
and \(x, y \in \mathbb{R}^n\).

\begin{enumerate}
\def\labelenumi{\alph{enumi}.}
\item
  Show that \(x-iy\) is also an eigenvector with eigenvalue
  \(\lambda - i \mu\).
\item
  Let \(V\) be the 2-dimensional subspace over \(\mathbb{R}\) spanned by
  \(x\) and \(y\). In other words, \(V\) is the set of linear
  combinations, with real coefficients, of the real-valued vectors \(x\)
  and \(y\). Show that \(V\) is an invariant subspace for \(A\), in the
  sense that for every \(z \in V\) we have \(Az
   \in V\).
\end{enumerate}

\subsubsection{Problem 5}\label{problem-5}

Do Problem 2.11.14 from the
\href{https://arxiv.org/pdf/2007.01367}{Official Notes}.

\subsubsection{Problem 6}\label{problem-6}

Consider the following matrix, whose exponential we derive in class:

\[
A = \begin{bmatrix} a & b \\ -b & a \end{bmatrix} \qquad a, b \in \mathbb{R}
\]

\begin{enumerate}
\def\labelenumi{\alph{enumi}.}
\tightlist
\item
  Re-derive an expression for \(e^{At}\) by diagonalizing \(A\) over
  \(\mathbb{C}\) and using the formula
  \[e^{a \pm ib}:= e^a \cos b \pm ie^a \sin b\] (treat this as the
  \emph{definition} of the complex exponential).
\item
  Using the formila \(e^{A(t+\sigma)} = e^{At}e^{A \sigma}\) and the
  result of (a), for suitably chosen values of \(a\) and \(b\), verify
  the trigonometric identities: \[
    \begin{align}
    \cos (t+\sigma) &= \cos t \cos \sigma - \sin t \sin \sigma \\
    \sin (t + \sigma) &= \sin t \cos \sigma + \cos t \sin \sigma
    \end{align}
    \]
\end{enumerate}
